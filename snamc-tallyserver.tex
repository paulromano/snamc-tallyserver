\documentclass{snamc2013}

%%%%%%%%%%%%%%%%%%%%%%%%%%%%%%%%%%%%%%%%%%%%%%%%%%%%%%%%%%%%%%%%%%%%%%%%%%%%%%%%
\usepackage{graphicx}  % allows inclusion of graphics
\usepackage{booktabs}  % nice rules (thick lines) for tables
\usepackage{microtype} % improves typography for PDF

\usepackage[breaklinks=true, linkcolor=black, citecolor=black, urlcolor=black]{hyperref}
\hypersetup{colorlinks=true,
  pdftitle={On the use of tally servers in Monte Carlo simulations of light-water
  reactors},
  pdfauthor={Paul K. Romano, Benoit Forget, Kord Smith, and Andrew Siegel}}
\usepackage[exponent-product=\cdot]{siunitx}

%%%%%%%%%%%%%%%%%%%%%%%%%%%%%%%%%%%%%%%%%%%%%%%%%%%%%%%%%%%%%%%%%%%%%%%%%%%%%%%%
\title{On the use of tally servers in Monte Carlo simulations of light-water
  reactors}

\author[1*]{Paul K. Romano}
\author[1]{Benoit Forget}
\author[1]{Kord Smith}
\author[2]{Andrew Siegel}

\affil[1]{Massachusetts Institute of Technology, Department of Nuclear Science
  and Engineering, 77 Massachusetts Avenue, Cambridge, MA 02139}
\affil[2]{Argonne National Laboratory, Theory and Computing Sciences, 9700 S
  Cass Ave., Argonne, IL 60439}
\affil[*]{\footnotesize\normalfont Corresponding Author, E-mail:
  paul.k.romano@gmail.com}

\abstract{An algorithm for decomposing tally data in Monte Carlo simulations
  using servers has recently been proposed and analyzed. In the present work, we
  make a number of refinements to a theoretical performance model of the tally
  server algorithm to better predict the performance of a realistic reactor
  simulation using Monte Carlo. The impact of subdividing fuel into annular
  segments on parameters of the performance model is evaluated and shown to
  result in a predicted overhead of less than 20\% for the BEAVRS PWR benchmark
  on the Mira Blue Gene/Q supercomputer. Additionally, a parameter space study
  was performed comparing implementations of the tally server algorithm in the
  OpenMC Monte Carlo code using blocking and non-blocking communication. The use
  of non-blocking communication was shown to reduce the communication overhead
  relative to blocking communication and can even result in negative overhead.}

\keywords{Monte Carlo, data decomposition, tally server, LWR, OpenMC}

\begin{document}
%%%%%%%%%%%%%%%%%%%%%%%%%%%%%%%%%%%%%%%%%%%%%%%%%%%%%%%%%%%%%%%%%%%%%%%%%%%%%%%%
\section{Introduction}

Typical parallel implementations of Monte Carlo particle transport rely on full
replication of the problem data on each process. This approach has been shown to
be highly scalable~\cite{ane-romano-2013}, but does not lend itself to problems
where the memory requirements exceed that of a single node. For the realistic
analysis of light-water reactors (LWRs), the memory requirements can be quite
severe. Neutron interaction cross sections, which need to be stored for over 400
nuclides at various temperatures, could consume up to 100 gigabytes of memory
for a practical simulation. For a depletion calculation, the required tally
memory could exceed 0.5 terabyte~\cite{jcp-siegel-2013}. A number of novel
algorithms may enable simulations involving continuous temperature distributions
to be performed using cross sections at 0 K~\cite{nse-yesilyurt-2012,
  nse-viitanen-2012}. Two methods of decomposing tally data, and thereby
reducing per-node memory requirements, have been proposed previously:
\emph{domain decomposition}~\cite{jcp-siegel-2012, jcp-siegel-2013} and
\emph{data decomposition}~\cite{trans-brown-2004}.

In a recent paper~\cite{jcp-romano-2013}, Romano et al. demonstrate an
implementation of data decomposition via a tally server algorithm and show that
it offers a viable means of performing full core light-water reactor simulations
via Monte Carlo. A theoretical model was developed to predict the performance of
a simulation using the tally server algorithm relative to a simulation based on
full memory replication. The model depends on a number of machine-, code-, and
problem-dependent parameters. In the present work, we revisit the derivation of
the expected performance of the tally server and make refinements to a number of
assumptions and parameters. The goal is to develop a more realistic expectation
for the performance of of the tally server algorithm when applied to simulation
of LWR problems.

%%%%%%%%%%%%%%%%%%%%%%%%%%%%%%%%%%%%%%%%%%%%%%%%%%%%%%%%%%%%%%%%%%%%%%%%%%%%%%%%
\section{Tally Server Model}
\label{sec:model}

During a Monte Carlo simulation, estimates of physical parameters, referred to
as \emph{tallies} are made by keeping running sums of scores from events such as
collisions or particle tracks. Normally, tallies are stored in local
memory. Synchronization between processors is typically performed only after
simulating a predetermined number of particles, referred to as a
\emph{batch}. However, since tally data is not needed for determining the random
walk of a particle, it can be stored remotely.

In the tally server algorithm, the tally data is stored in the address space of
a process whose sole purpose is to receive scores from other processes (which we
call \emph{tally servers}) and increment the tallies accordingly. Thus, a total
of $p$ processes are divided into $c$ compute processes and $s$ tally
servers. Each of the compute processes is assigned a set of particles that it
will track. As particle tracking is simulated, an array of scores is sent to a
tally server at each event that results in a contribution to a tally. Since all
tally accumulation is performed on the server, the compute processes do not need
to store the tallies in memory (other than meta-data describing the tally).

The goal of the analysis here is to develop a model for for the expected time to
simulate $N$ particles using the tally server algorithm relative to a simulation
with no data decomposition. To that end, we first define a number of parameters:
\begin{align*}
  \mu &= \text{average time to simulate a single particle}, \\
  f &= \text{number of tallying events per particle}, \\
  d &= \text{bytes of tally data sent at each event}, \\
  \alpha &= \text{effective application-level network latency}, \\
  \beta &= \text{effective application-level inverse network bandwidth}
\end{align*}
The latency and inverse bandwidth are determined by the network interconnect;
$f$, $\mu$, and $d$ will depend on the machine hardware as well as the code
being used and the model being simulated. Thus, while these parameters may be
hard to predict, they can easily be measured from an actual simulation. Once
these parameters are known, we can develop a rough estimate for the
time-to-solution with and without tally servers. In a normal simulation without
tally servers, the expected time to simulate $N$ particles is, assuming perfect
parallel scaling,
\begin{equation}
  \label{eq:time-without}
  t_0 = \frac{N\mu}{p}.
\end{equation}
When the tally server algorithm is used, there are two sources that lead to of
overhead: 1) availability of fewer processors to simulate particles and 2)
network communication for tally data from compute processes to the servers. The
expected simulation time when using tally servers is identical to the expression
in \autoref{eq:time-without} but with $p$ replaced by $c$:
\begin{equation}
  \label{eq:time-with}
  t_c = \frac{N\mu}{c}.
\end{equation}
Since $f(\alpha + d \beta)$ is the expected tally server communication time for
one particle and $N/c$ is the number of particles per processor, the total
expected communication time is
\begin{equation}
  \label{eq:time-send}
  t_s = \frac{fN}{c} \left ( \alpha + d\beta \right ).
\end{equation}
We then define $t = t_c + t_s$ as the total simulation time using tally
servers. Combining \autoref{eq:time-without}, \autoref{eq:time-with}, and
\autoref{eq:time-send}, we obtain an expression relating the simulation time with
and without tally servers
\begin{equation}
  \label{eq:relation}
  \frac{t}{t_0} = \frac{p}{c} \left [ 1 + \frac{f}{\mu} \left ( \alpha + d\beta
  \right ) \right ].
\end{equation}
The first factor on the right-hand side of \autoref{eq:relation}, $p/c$
represents the loss in efficiency due to having fewer processes tracking
particles. The remaining term within the square brackets represents the loss in
efficiency due to the necessary network communication. In this work, we will
primarily be concerned with the communication overhead,
\begin{equation}
  \label{eq:overhead}
  \Delta_s = \frac{f}{\mu} \left ( \alpha + d\beta \right ).
\end{equation}

%%%%%%%%%%%%%%%%%%%%%%%%%%%%%%%%%%%%%%%%%%%%%%%%%%%%%%%%%%%%%%%%%%%%%%%%%%%%%%%%
\subsection{Model Refinements}

In the previous work by Romano et al.~\cite{jcp-romano-2013}, estimates of the
tally server parameters were made by analyzing a hypothetical depletion
simulation of the Monte Carlo Performance Benchmark~\cite{mc-hoogenboom-2011} on
two target supercomputers: the Titan Cray XK7 at Oak Ridge National Laboratory
and Intrepid Blue Gene/P at Argonne National Laboratory. For the sake of
simplicity, some of the assumptions made in estimating these parameters were not
conservative. We now re-visit those assumptions to develop more realistic
estimates to determine what effect, if any, they have on the expected
performance of the tally server algorithm on a modern supercomputer.

\subsubsection{Target Model}

Rather than look at the Monte Carlo Performance Benchmark, which contains many
unrealistic simplifications (e.g., no fuel enrichment zoning and no control
rods), we have chosen as our target problem the BEAVRS PWR benchmark
model~\cite{mc-horelik-2013}. This model includes accurate enrichment loadings,
burnable absorber patterns, and control bank positions as well as
faithfully-modeled axial grid spacers, core baffle structures, neutron shield
panel structures, and relevant core internals. The use of a different model will
have an impact on $\mu$ and $f$. For Mira Blue Gene/Q, the particle tracking
rate for the BEAVRS benchmark is about $1/\mu = \SI{69}{particle/\second}$. This
very similar to the particle tracking rate for the Monte Carlo performance
benchmark on Blue Gene/P, and the number of tracks in fuel is virtually the same
at $f = 21$. Assuming the same physical quantities need to be tallied, $d$ will
not change. As in our previous work~\cite{jcp-romano-2013}, a range of $d$ will
be investigated.

\subsubsection{Annular Regions in Fuel}

In a depletion simulation, six reaction rates for each nuclide must be tallied
each time a particle track crosses fuel. Furthermore, it is necessary to
subdivide fuel regions into annular segments since spatial self-shielding will
result in the outer part of a fuel pin depleting faster than the inner part. The
impact of this subdivision of the fuel on $f$ and $\mu$ has not previously been
accounted for. With an increasing number of subdivisions, the number of events
that will result in contributions to tallies will increase. At the same time,
the time to simulate a single particle will increase since there will be more
surface crossings, re-evaluation of cross sections, tallying events, etc.

To explicitly determine the effect of fuel subdivision on $f$ and $\mu$, a
series of simulations were run using the OpenMC Monte Carlo
code~\cite{ane-romano-2013} on the BEAVRS PWR benchmark
model~\cite{mc-horelik-2013} varying the number of annular regions in the fuel
from 1 to 10.  \autoref{fig:events} shows the dependence of $f$ on the number of
annular regions. While not intuitively obvious \emph{a priori}, this figure
demonstrates that the number of tracks in fuel is directly proportional to the
number of annular regions.
\begin{figure}[htb]
  \centering
  \includegraphics[width=3.0in]{events}
  \caption{Number of tracks in fuel as a function of the number of annular
    regions.}
  \label{fig:events}
\end{figure}

In OpenMC, each time a particle enters a new material, the macroscopic cross
sections must be calculated. This is true even if the particle hasn't changed
energy. Thus, as the number of annular segments in fuel increases, the
calculation time will increase due predominantly to the extra cross section
evaluations. \autoref{fig:time} shows the dependence of $\mu$ on the number of
annular regions as measured by the OpenMC Monte Carlo code on an Intel Core i5
Processor. While the relative simulation time time also appears to increase
linearly with the number of annular regions, it is not directly proportional
like the number of tracks since only a fraction of the simulation time is spent
tracking particles in fuel.
\begin{figure}[b]
  \centering
  \includegraphics[width=3.0in]{time}
  \caption{Relative simulation time per particle as a function of the number of
    annular regions.}
  \label{fig:time}
\end{figure}

\subsubsection{Network Interconnect}

For the present analysis, rather than looking at the Titan Cray XK7 or Intrepid
Blue Gene/P supercomputers, our target architecture is the Mira Blue Gene/Q
supercomputer at Argonne National Laboratory. Mira has 48 racks, each with 1024
nodes containing a 16-core PowerPC A2 processor for a total of 768,432 processor
cores. More importantly, the Blue Gene/Q network interconnect utilizes a 5D
torus and has lower latency and high bandwidth than the interconnect used for
Blue Gene/P. The nearest-neighbor MPI latency has been observed to be about
\SI{2.0}{\us}~\cite{hammond-2012} and the maximum-hop latency is about
\SI{3.0}{\us}~\cite{kumaran-2012}. Rather than use either one of these figures,
we will assume an average latency of $\alpha = \SI{2.5}{\us}$. The internode
single link bandwidth is about
\SI{1.8}{\giga\byte/\second}~\cite{kumaran-2012}. Consequently, we will use
$\beta = \SI{5.55e-10}{\second/byte}$. \autoref{tab:parameters} gives a summary
of the parameters used in the model predictions for the tally server overhead as
well as those used in our previous work~\cite{jcp-romano-2013}.
\begin{table}[htb]
  \caption{Parameters used for tally server overhead model}
  \label{tab:parameters}
  \begin{tabular}{ c l c c }
    \toprule
    Parameter & Description & Intrepid & Mira \\
    \midrule
    $\alpha$ & Latency (s) & \num{3.53e-6} & \num{2.5e-6} \\
    $\beta$ & Bandwidth (s/byte) & \num{2.60e-9} & \num{5.55e-10} \\
    $1/\mu$ & Particles/second & 76 & 69 \\
    $d$ & Data/event (bytes) & 0 -- 15,360 & 0 -- 15,360 \\
    $f$ & Events/particle & 21 & 21--213 \\
    \bottomrule
  \end{tabular}
\end{table}

\subsubsection{Predicted Overhead}

As discussed earlier, the increase in simulation time when using tally servers
can be attributed to 1) having fewer processors tracking particles and 2)
network communication. The first factor that increases the simulation time is
known and is simply determined by the user's choice of $p$, $c$, and $s$. Thus
we will evaluate only the overhead from network communication as given in
\autoref{eq:overhead}.

In the previous section, we demonstrated that when subdividing the fuel pins
into annular regions, the number of tallying events per particle $f$ is directly
proportional to the number of annular regions, whereas $\mu$ increases only
slightly. Thus, the communication overhead based on \eqref{eq:overhead} will
increase almost in direct proportion to the number of annular
regions. \autoref{fig:model} shows the predicted overhead on the Mira
supercomputer as a function of $d$ for varying numbers of annular regions based
on the results in \autoref{fig:events} and \autoref{fig:time}. The upper limit
on $d$ is 15,360 bytes, the amount of tally data for six reaction rates in each
of 320 nuclides within a material. The latency and bandwidth of the interconnect
were taken from \autoref{tab:parameters}. Even when 10 annular regions in the
fuel are modeled, the maximum predicted communication overhead is still under
20\%.
\begin{figure}[htb]
  \centering
  \includegraphics[width=3.3in]{model}
  \caption{Estimated tally server overhead on the Mira Blue Gene/Q supercomputer
    as a function of the number of annular regions.}
  \label{fig:model}
\end{figure}

\subsubsection{End-of-batch Accumulation}
\label{sec:end-of-batch}

One aspect of the algorithm that was not previously accounted for in the model
of overhead is the accumulation of tallies at the end of a batch. For
statistical purposes, after a set of $N$ neutrons are simulated, the accumulated
score for each tally random variable is added to a running sum, and the square
of the accumulated value is added to a sum of squares. These sums enable the
sample variance to be calculated at the end of the simulation. When a tally
server algorithm is used, the task of incrementing these two sums is shifted
from the compute processors to the servers. Said another way, the total amount
of work the compute processes must perform is reduced slightly. As a result, the
reduced work can partially or completely negate the network communication
overhead.

To model this effect, we break up the average time to simulate $N$ particles
into two components, $N\mu = N\mu_t + \mu_b$, where $\mu_t$ is the average time
to transport a particle and $\mu_b$ is the average time to calculate sums and
sums-of-squares. Since $\mu_b$ is directly proportional to the total number of
tally scores, which in turn is typically proportional to $d$, we can express it
as $\mu_b = \mu_b' d$. Without tally servers, the total time to simulate $N$
particles on $p$ processors becomes
\begin{equation}
  \label{eq:time-without-mod}
  t_0 = \frac{N\mu_t + \mu_b' d}{p}.
\end{equation}
When tally servers are used, the time spent incrementing the sums is offloaded
to the servers. Thus, the total tracking time on $c$ compute processes is
\begin{equation}
  \label{eq:time-with-mod}
  t_c = \frac{N\mu_t}{c}
\end{equation}
As before, when we combine \autoref{eq:time-without-mod},
\autoref{eq:time-with-mod}, and \autoref{eq:time-send}, we obtain an expression
relating the simulation time with and without tally servers:
\begin{equation}
  \label{eq:relation-mod}
  \frac{t}{t_0} = \frac{p}{c} \left [ \frac{\mu_t + f \left ( \alpha + d\beta
      \right )}{\mu_t + \frac{\mu_b' d}{N}} \right ].
\end{equation}
The communication overhead, defined earlier as the bracketed term minus unity,
now includes a term in the denominator that will increase with $d$:
\begin{equation}
  \label{eq:overhead-mod}
  \Delta_s = \frac{\mu_t + f \left ( \alpha + d\beta \right )}{\mu_t +
    \frac{\mu_b' d}{N}} - 1.
\end{equation}
According to \autoref{eq:overhead-mod}, it is possible for the communication
overhead to be negative if $Nf ( \alpha + d\beta ) < \mu_b' d$. If $d$ is
sufficiently large that the latency is negligible ($\alpha \approx 0$), then the
condition for negative overhead becomes $Nf\beta < \mu_b'$. While this condition
no longer depends on $d$, $\mu_b'$ can still increase if the total number of
tally score bins is increased (e.g., by refining a mesh over which scores are
being tallied). \autoref{fig:model-negative} shows the predicted overhead on the
Mira supercomputer as a function of $d$ for the original model in
\autoref{eq:overhead} and the modified model in \autoref{eq:overhead-mod}. The
parameters $\mu_t$, $\alpha$, and $\beta$ are all from \autoref{tab:parameters}
and it was assumed that $\mu_b'/N = \SI{50}{\nano\second/byte}$. This value was
chosen merely to demonstrate that negative overhead is possible and that
$\mu_b'$ need not be exceedingly large. For small values of $d$, the overhead is
dominated by the latency term. For larger values of $d$, \autoref{eq:overhead}
results in an increasing overhead due to the bandwidth term whereas
\autoref{eq:overhead-mod} results in decreasing overhead since $Nf (\alpha +
d\beta) < \mu_b' d$.
\begin{figure}[htb]
  \centering
  \includegraphics[width=3.3in]{model_negative}
  \caption{Estimated tally server overhead accounting for accumulation.}
  \label{fig:model-negative}
\end{figure}

To summarize, there are two key takeaways:
\begin{enumerate}
\item Negative overhead is possible due to offloading the incrementing of tally
  sums and sums-of-squares to the tally servers and is more likely to occur when
  a large number of quantities are being tallied.
\item In practice, the beneficial effect of offloading this operation may be
  masked by large $N$. Particularly in reactor simulations where it is expected
  that a single batch of neutrons may exceed one billion neutrons, it is
  unlikely that negative overhead could be achieved.
\end{enumerate}

The foregoing analysis has thus far assumed that network communication is
blocking. However, if non-blocking communication is used, the communication
operations may overlap with computation. In the best case scenario, the
non-blocking sends from compute processes would return instantaneously implying
that $t_s = 0$. This in turn would imply that
\begin{equation}
  \label{eq:overhead-nb}
  \Delta_s = \frac{\mu_t}{\mu_t + \frac{\mu_b' d}{N}} - 1 = -\frac{\mu_b'
    d}{N\mu_t + \mu_b' d}.
\end{equation}
We see here that with non-blocking communication, negative overhead is possible
regardless of the size of $N$.

While non-blocking communication may reduce the network communication overhead
at the sender to a level that is negligible or even negative, it's important to
keep in mind that the time to complete a batch of neutrons is still limited by
the lesser of the time the compute processes require to transport the particles
and the time the tally servers require to accumulate tallies. The latter time is
constrained in the sense that an excessively large support ratio, $c/s$, would
result in network contention at the tally servers. For the tally server, there
is no computation to be performed and thus no opportunity to overlap
communication and computation---handling communication is the sole purpose of
the server. In light of this, the latency and bandwidth of the network are still
crucial parameters that have a bearing on the feasibility of the tally server
algorithm.

%%%%%%%%%%%%%%%%%%%%%%%%%%%%%%%%%%%%%%%%%%%%%%%%%%%%%%%%%%%%%%%%%%%%%%%%%%%%%%%%
\section{Results}
\label{sec:results}

A complete implementation of the tally server algorithm in the OpenMC Monte
Carlo code was previously described by Romano et al.~\cite{jcp-romano-2013}. The
initial implementation, which was based on blocking communication, was tested
over a wide range of parameters on two supercomputers: the Titan Cray XK7
supercomputer at ORNL and the Intrepid Blue Gene/P supercomputer at ANL. It was
argued based on the performance model that in the limit of an optimal support
ratio, the use of non-blocking communication could reduce the total overhead by
a factor of two, but such an implementation was never tested. Since then, a
tally server algorithm based on non-blocking communication has been implemented
in a branch of OpenMC.

The performance model developed in \autoref{sec:model} depends on a variety of
parameters. For our target system, the Mira Blue Gene/Q supercomputer, $\alpha$,
$\beta$, and $\mu$ are constant and can be determined based on measured data as
previously discussed. The remaining parameters are manipulated by varying the
definition of the tallies and the job parameters. To fully test the performance
of the non-blocking tally server implementation, a parameter study was performed
that covers a range of the parameters $p$, $s$, and $d$. For the present work,
we have focused specifically on the dependence of the communication overhead on
$d$ for varying support ratios $c/s$, total number of processors $p$, and a
fixed $f$. As expressed in \autoref{eq:overhead-mod}, we do not expect the
overhead to vary with either the support ratio or the total number of
processors---nevertheless we have chosen to include them as parameters since any
limitation to the scalability of the algorithm is likely to show up as a trend
with $p$ or $c/s$.

To begin, a number of baseline simulations of the BEAVRS benchmark were run
without tally servers to determine the dependence of $\mu_t$ on $d$. These
simulations were run on Mira with 16 processors and a total of 32,000 particles
per batch. Ten batches were run both without tallies (referred to as
\emph{inactive batches}) and with tallies (\emph{active batches}). For each
case, a tally was set up with a mesh filter and a second filter to match only
events within the fuel volume. Six reaction rates were tallied for varying
numbers of nuclides, starting with 5 nuclides and doubling the number of
nuclides up to 320. Thus, the amount of data sent at each event varied from 240
bytes up to 15.36 kilobytes. \autoref{fig:mu-d} shows the observed dependence of
$\mu_t$ on $d$ normalized to the $d=5$ case.
\begin{figure}
  \centering
  \includegraphics[width=3.3in]{baseline}
  \caption{Observed dependence of $\mu_t$ on the amount of data tallied, $d$, on
    Mira.}
  \label{fig:mu-d}
\end{figure}

The parameter study using tally servers on the Intrepid supercomputer consisted
of two sets of 168 simulations with each combination of the following
parameters: $p = 16,32,64,128,256,512$, $c/s = 1,3,7,15$, and $d = 240, 480,
960, 1920, 3840, 7680, 15360$. The first set was performed with blocking
communication between the compute processes and the servers and the second set
with non-blocking communication. Like the baseline cases, the runs with tally
servers had 10 inactive batches, 10 active batches, and $N/p = 500$. The
effective overhead from tally servers was determined in the following
manner. First, the expected overhead due to looking up cross sections during
tallying was subtracted from the active batch time based on the results from the
baseline cases. Then, the adjusted simulation time in active batches was divided
by the inactive batch time to determine the overhead in active batches. The
result is a quantity that is a proxy for the communication overhead,
$\Delta_s$. One should take note that it does not account for the fact that we
have fewer compute processes.  The overhead calculated in this manner for $c/s =
1$, $c/s = 3$, $c/s = 7$, and $c/s = 15$ is shown in \autoref{fig:results-mira-r1},
\autoref{fig:results-mira-r3}, \autoref{fig:results-mira-r7}, and
\autoref{fig:results-mira-r15}, respectively.
\begin{figure}[htb]
  \centering
  \includegraphics[width=3.3in]{mira_r1}
  \caption{Observed tally server overhead on ANL Mira with 1 compute process per
    server.}
  \label{fig:results-mira-r1}
\end{figure}
\begin{figure}[htb]
  \centering
  \includegraphics[width=3.3in]{mira_r3}
  \caption{Observed tally server overhead on ANL Mira with 3 compute process per
    server.}
  \label{fig:results-mira-r3}
\end{figure}
\begin{figure}[htb]
  \centering
  \includegraphics[width=3.3in]{mira_r7}
  \caption{Observed tally server overhead on ANL Mira with 7 compute process per
    server.}
  \label{fig:results-mira-r7}
\end{figure}
\begin{figure}[htb]
  \centering
  \includegraphics[width=3.3in]{mira_r15}
  \caption{Observed tally server overhead on ANL Mira with 15 compute process per
    server.}
  \label{fig:results-mira-r15}
\end{figure}

Compared to our previous study, the observed communication overhead is lower for
large $d$ primarily due to the higher bandwidth on Mira compared to Titan or
Intrepid. In all cases, the communication overhead is less than 6\%, whereas for
Intrepid and Titan it had exceeded 30\% in some cases. A more striking feature
in all the results is the fact that all non-blocking cases exhibit a clear trend
of increasingly negative overhead for large $d$. Based on the discussion in
\autoref{sec:end-of-batch}, this is a direct consequence of the fact that the
incrementing of tally sums and sums-of-squares has been offloaded to the tally
servers. Had the choice of $N$ been larger, this effect would have been
mitigated. That negative overhead could be observed at all is a testament to the
inherently fast network interconnect on Mira resulting in little overhead,
especially when non-blocking semantics are used.

It is also of interest to observe the behavior of the tally server overhead with
increasing numbers of total processors. According to the performance model, the
overhead should not depend on the number of processors
used. \autoref{fig:results-mira-cs} shows the overhead plotted as a function of
$p$ for cases with $d=15360$. For the simulations where blocking communication
was used, there is no clear trend with $p$. The overhead when using 16, 64, and
128 total processor cores was consistently positive whereas the overhead turned
negative for 32, 256, and 512 total processors. Despite the odd behavior with
changes in $p$, there was little variation as a function of the support ratio,
$c/s$. When non-blocking communication was used, the overhead was consistently
negative for all cases.
\begin{figure}[htb]
  \centering
  \includegraphics[width=3.3in]{mira_cs}
  \caption{Observed tally server overhead on ANL Mira as a function of $p$ with
    $d=15360$.}
  \label{fig:results-mira-cs}
\end{figure}

%%%%%%%%%%%%%%%%%%%%%%%%%%%%%%%%%%%%%%%%%%%%%%%%%%%%%%%%%%%%%%%%%%%%%%%%%%%%%%%%
\section{Conclusions}

In the present work, we have made further inroads towards evaluating the
potential for the tally server data decomposition algorithm to be applied to
Monte Carlo simulations of light-water reactors. The two major contributions are
1) improvements in the theoretical performance model, and 2) a thorough
parameter space study looking at the impact of blocking vs. non-blocking
communication in a real tally server implementation in the OpenMC Monte Carlo
code.

In our previous work, the Monte Carlo performance benchmark, otherwise known as
the Hoogenboom-Martin benchmark, was used as the basis for evaluating tally
server performance model parameters. That benchmark model was overly simplified,
and the recent introduction of a more realistic PWR benchmark, BEAVRS, has
allowed us to re-evaluate the model parameters. The change of benchmark models
did not have a significant effect on any model parameters. The simplified
theoretical model that was developed previously has also been refined to better
predict the performance of a realistic reactor simulation. Most importantly, the
effect of fuel subdivision on the number of particle tracks and calculation rate
for the BEAVRS benchmark was quantified using OpenMC. It was shown that the
predicted overhead due to tally servers increases linearly with the number of
annular regions in fuel. Nevertheless, even with 10 regions, the predicted
overhead of using tally servers is less than 20\% on the Mira supercomputer over
a wide parameter regime. Thus, the subdivision of fuel pins into unique
depletion regions should not be a major impediment towards achieving
high-fidelity simulations that rely on tally servers.

A modified implementation of the tally server algorithm in OpenMC using
non-blocking communication was tested on the Mira supercomputer along with the
original implementation based on blocking communication. The observed
communication overhead was reduced when using non-blocking communication as
previously predicted. Furthermore, the communication overhead decreased to the
point that it was negative as the amount of data being sent at each tally event
increased. This was attributed to the accumulation of tally scores at the end of
a statistical batch being offloaded to the tally servers rather than being
performed by the compute processes. It is important to recognize that the
negative overhead observed is a consequence of the particular choice of run
parameters and would be unlikely to occur in a hypothetical reactor depletion
simulation where the total number of particles per statistical batch is
necessarily very large, thus reducing the important of any end-of-batch
operations.

The basic conclusions of our previous work, i.e., that the tally server
algorithm is a successful approach to circumventing on-node memory constraints
associated with detailed Monte Carlo reactor simulations, in unchanged in light
of the evidence presented in this work. While the tally server algorithm could
already be employed on the world's fastest supercomputers today, the need for an
extremely fast network interconnect means that it may not be amenable for use on
commodity computer architectures that would more likely be used by scientists
for day-to-day work.

%%%%%%%%%%%%%%%%%%%%%%%%%%%%%%%%%%%%%%%%%%%%%%%%%%%%%%%%%%%%%%%%%%%%%%%%%%%%%%%%
\section*{Acknowledgments}

This research was performed under appointment of the first author to the
Rickover Fellowship Program in Nuclear Engineering sponsored by Naval Reactors
Division of the U.S. Department of Energy. This work was also supported in part
by the Office of Advanced Scientific Computing Research, Office of Science,
U.S. Department of Energy, under Contract DE-AC02-06CH11357.

%%%%%%%%%%%%%%%%%%%%%%%%%%%%%%%%%%%%%%%%%%%%%%%%%%%%%%%%%%%%%%%%%%%%%%%%%%%%%%%%
\bibliographystyle{ans}
\bibliography{references}
\end{document}
